There are several existing module systems that have been proposed or
are in use for Java and other similar languages. 
Several \cite{javajars, OSGi4, netassemblies} have been in use in industry 
applications for several years now, and common operations on modules
can be derived from their usage.

\SubSection{Composition}

In Java, the classpath provides the most basic level of module composition
for applications. The classpath provides a list of libraries that are
available during compilation or execution. Classes are looked up in the 
classpath entries in the order they appear. This is fine as long as the sets 
of classes that the entries provide are disjoint. However, if a class can 
be found in multiple entires, Java will always look up that class in the 
first entry on the classpath, which may not be what the developer intends. 
This is JAR hell, and the most common error caused by this is loading 
the incorrect class when there is more than one version of a library in the classpath.

Java module systems have more or less been built as solutions to JAR hell. 
OSGi \cite{OSGi4} bundles, JSR 277 modules \cite{JSR277} and .NET assemblies \cite{netassemblies}
use constraints to make sure that the modules used by an application's class satisfy the 
class' requirements. 

However, the current solutions are still not completely satisfactory when there are still multiple
types that satisfy a type reference even with version constraints. In the case
of OSGi, this occurs as split packages \cite{iJAMComments}, when a package is both contained by
two visible modules, modules that are not necessarily versions of each other
and hence not limited by the version import constraints.

\SubSection{Versioning}
Adding versions as metadata to modules is the common solution to multiple
versions of a class. OSGi bundles, the JSR 277 Java Module System
proposal and .NET assemblies 
have extensive mechanisms for allowing a module's client to specify the
versions of a module that it supports.

Higher number versions are expected to be backward compatible. If this is not
the case, the module provider's only resort is to note this in separate documentation,
and rely on the module's clients to properly update their version constraints
to reflect this information. The same applies for a module's features. A module's
supported features are usually documented separately and are linked to the version
of the module.

\SubSection{Imports and Exports}

The set of a module's imports and exports specify the packages (or types) that
the module requires and provides, respectively. This affects class
visibility outside the module and is used to enhance information hiding. 

The most basic form of export is the {\tt public} modifier in Java. Only
public modified classes, methods or fields are available outside a java package.

OSGi bundles and JSR 277 modules allow a user to specify a set of package imports 
and exports that specifies the modules required packages, and the set of packages 
it makes available outside the module. They also both allow a module
to explicitly import another bundle/module and its packages.

The JSR 294\cite{JSR294} proposal introduces the {\tt module} modifier, that
extends visibility to extend to the entire module, which spans multiple java
packages.

Component Nextgen \cite{componentnextgen} uses interfaces for modules, which
specify a parameterized signature of classes that a module provides.

\SubSection{Instantiation}
With increasingly large and complex systems that involve a large number of components,
each possibly requiring different versions of a common module,
the ability to allow multiple versions of the same class to coexist become necessary.
Indeed, a single module may require two versions of a another module because its
components require different versions. This implies the creation of multiple instances 
of a module in a repository from which the classes are to be loaded, and the ability
for a module to load and distinguish multiple versions of that module in its code.

iJAM \cite{iJAM} provides a module instantiation feature that allows a module to create
separate instances of its imported modules. This was a solution for type lookup
problems caused by the {\it parent-then-self} class loading scheme proposed for
JSR 294.

However, type lookup when there are multiple instances of the same module are still
reliant of fixed lookup orders: {\it self-then-parent} for iJAM and the reverse for JSR 294.
Real world applications require more flexibility than is provided by a single fixed
lookup rule \cite{iJAMComments}.