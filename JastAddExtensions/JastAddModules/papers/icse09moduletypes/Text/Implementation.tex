The type system just described has been implemented in a
module-aware Java Compiler coded in JastAdd \cite{jastadd},
an extensible aspect-oriented compiler framework. This forms
only the compile-time component of the module system. However,
we do not foresee any serious design faults that would preclude
the implementation of a run-time system using classloaders.

Module instantiation was implemented by mangling
the package names of the member classes for each instance of 
a module. An \texttt{import own} generated a name prefixed by 
the importing instance's name, thereby ensuring the separation
of the namespaces of different instances of the same module.
The type lookups within the program were then pointed to the
name-mangled versions of the class to which they should resolve.

The current implementation is only the compile-time side of
the system, and simulates run-time module loading by generating 
name-mangled versions of the instantiated modles. 
The run-time component will be implemented through
classloaders, and is left as future work.

The implementation and the JHotdraw 7.1 case study are available at : 
\textit{http://progtools.comlab.ox.ac.uk/members/neil}


%Implemented as name mangling and changes to lookup.

%\SubSection{Name mangling and lookup}

%\SubSection{Subtyping and Shadowing}

%\SubSection{Module Interfaces and Exports}

%\SubSection{Limitations}

%Needs all sources at compile time. But can be resolved once
%classloader implementation is present.

%Needs to be told which is the instance module. But this is
%more a constraint added due to resource and time limitations.

%\SubSection{Future Implementation as ClassLoader}

%Mention possible issues with local module qualified imports. However,
%these will probably be solved by OSGi's one class loader per 
%bundle instance solution.