The type system just described is very flexible, and not only does it
allow us to model current operations and constraints on modules, it also
allows the expression of new constraints using the type system. The following
sections go into more detail on how the type system can be used.

\SubSection{Composition}

A module's classpath is modeled by its module imports. Imports also have the advantage
of allowing a choice in which class to load, and in causing an
ambiguous type error  when more than one bundle offers
a class instead of just loading (or failing to load) the wrong class at runtime.

As mentioned previously, module-less classes are considered to be part 
of the ``default module'', and its singleton instance is implicitly 
imported by all modules. Completely module-less builds also work as before.

Special lookups such as the bootclasspath and java.lang classes can be
modeled as being members of special modules (bootmodule and javalib), 
which are given preference during lookup (bootmodule first, then javalib,
then everything else using the lookup rules for modules). Membership
in these special modules will probably be implemented in a similar way to
the OSGi \texttt{extends} feature.

Split packages become less of an issue, as they can be detected at
build time and can be avoided by using module qualified type references.

Errors in composition would now occur at the module level, instead of
class cast or invalid method invocation exceptions. This would make it
easier to identify and debug.

\SubSection{Versioning}

Overriding can be used to model versioning. It is already done
implicitly by current usage of existing module systems: newer versions  
override past versions as long as they are backward compatible.
However, the type system allows additional constraints to be specified using
types. Non-backward compatibility can be modeled by not specifying a
previous version in the overrides list.

\begin{lstlisting}[caption=Versioning Using Overrides]
//file appv1_1.module
module appv1_1;

//file appv1_2.module
//version 1.2 is backward compatible to 1.1
module appv1_2 overrides appv1_1;

//file appv2_0.module
//but version 2 is no longer backward compatible
module appv2_0; 
\end{lstlisting}

In this manner backward compatibility can now be expressed in 
and checked by the type system, instead of just being written
down in separate documentation.

Extends allows specialization and patch releases that do not contain 
the entire previous version's classes, as already demonstrated in
the previous section. However, determining when to use module
extension over existing extension patterns using classes warrant
further study.

Interfaces can also be used to model version ranges, and even feature
compatibility across versions. Versions that are in the same version 
range or are feature compatible implement the same interface.

\begin{lstlisting}[caption=Interfaces as Properties]
//file appv1.module
module appv1 implements appv1to2, supportsfeature1;

//file appv2.module
module appv2 implements appv1to2, appv2to4, 
	supportsfeature1;

//file appv3.module
module appv3 implements appv2to4, 
	supportsfeature1, supportsfeature2; //new feature

//file appv4.module
module appv4 implements appv2to4, 
	supportsfeature2; //no longer supports feature1
\end{lstlisting}

In general, interfaces can be used as a tag for any arbitrary constraint, 
in a similar manner that empty interfaces are used in Java to mark classes 
that satisfy a certain property.

It should be noted that extends, implements and overrides are all orthogonal,
so a module can be a subtype of another module, override a set of modules and
implement a set of interfaces. This allows a module author to reuse portions
of the supertype module, specify certain properties that the module satisfies
using module interfaces, and specify a set of modules it overrides (which may
include its supertypes, even though a subtype already implicitly overrides
its supertype).

These extends, implements and overrides declarations do not have to be explicit.
Common usage of existing module systems mostly already assume a that a 
module overrides another module of the same name if its version 
number is higher. Existing module systems can generate these version-related types 
internally, while still allowing explicit declarations for patch releases 
using extends and using module interfaces for expressing non-version
related constraints. Alternatively, weak interfaces can be modified to include
versions as part of its constraints, instead of just the set of exposed packages.

As shown on the previous section, merge and replace can be used to 
update references to old versions of a module with a newer version.
Having \textbf{own} instances together with merges and replace allows
fine-grained definition of the instance equalities of a module's dependencies.

Merge and replace, however, are imperative operations, and would require
some sort of debugging environment when errors arise. Given their flexibility,
there are many anti-patterns for the use of merge and replace, one of which
is directly replacing a merged member instead of using the merged reference.
Developing patterns and identifying anti-patterns for these operations will be
crucial for the successful deployment of the type system for real-world applications.

\SubSection{Imports and Exports}

Imports and exports are directly implemented by module imports and export package.
OSGi's loose coupling using export and import packages can be simulated by
the module system internally generating a weak interface that satisfies the import
package specification, finding a visible module that matches that weak interface, and
inserting an import and replace module declaration in the module that contained the import package
declaration.

Packages, however, are still just class modifiers. A possibly useful addition to 
\texttt{package-info.java} is an annotation \texttt{@RequiredClass("classname")}
that checks the existence of a class in a package.

\SubSection{Instantiation}

Import own is derived directly from iJAM, and is improved with an explicit exporting 
mechanism and a module lookup system using module qualifiers. Singleton module instances mirror
OSGi singletons. The module qualifier in type references and
imports allow fine-grained selection of the class to load in contrast
to fixed loading strategies. They also allow for module lookups relative to
a module's context, decoupling modules from globally named module instances.

However, instantiation as currently implemented is still done eagerly, pulling in all
of an imports dependencies the moment the import is instantiated. This may not be necessary,
especially when interfaces are used. A way to instantiate a module and its dependencies 
lazily would be required to reduce the number of dependencies, and is left as future work.

\SubSection{Further Implications}

Note that the syntax only represents the type system, it is not the only way
to implement it. As mentioned, overriding is already implicitly implemented
by versions. Other existing features such as loose coupling in OSGi can
also be modeled, i.e. by using weak interfaces and a generated import and replace
declaration. The type system is also not limited to
the current implementation. For example, the import declaration can be 
extended to use OSGi's extensive constraint features for packages and bundles.

Separate compilation of modules with module interface dependencies would require
the ability to compile and lookup module qualified names during runtime. We believe
it is possible to do so with a classloader implementation of the module system. Reflection
APIs would also require the context in which the lookup should be done in addition to
the class' package and name. These are left as future work.

The overrides declaration creates synthetic types for the overridden modules 
if they are not present in the compilation. These types are currently very loose,
in that they can be assigned to a module reference of any type. Tightening the
constraints on synthetic types using information from overriding types is left
as future work.
