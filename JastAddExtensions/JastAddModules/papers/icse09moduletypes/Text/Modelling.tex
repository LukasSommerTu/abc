The type system just described is very flexible, and not only does it
allow us to model current operations and constraints on modules, it also
allows the expression of new constraints using the type system. The following
sections go into more detail on how the type system can be used.

\SubSection{Composition}

A module's classpath is modeled by its module imports. Imports also have the advantage
of allowing a choice in which class to load, and in causing an
ambiguous type error  when more than one bundle offers
a class instead of just loading (or failing to load) the wrong class at runtime.

As mentioned previously, module-less classes are considered to be part 
of the ``default module'', and its singleton instance is implicitly 
imported by all modules. Completely module-less builds also work as before.

Special lookups such as the bootclasspath and java.lang classes can be
modeled as being members of special modules (bootmodule and javalib), 
which are given preference during lookup (bootmodule first, then javalib,
then everything else using the lookup rules for modules). Membership
in these special modules will probably be implemented in a similar way to
the OSGi \texttt{extends} feature.

Split packages become less of an issue, as they can be detected at
build time. Class space consistency (in OSGi) can now be violated,
as long as ambiguous type references are resolved using module qualifiers.

\SubSection{Versioning}

Subtyping and overriding can be used to model versioning. It is already done
implicitly by current usage of existing module systems: newer versions are 
subtypes of (or override) past versions as long as they are backward compatible.
However, the type system allows additional constraints to be specified using
types. Non-backward compatibility can be modeled as changing to a
completely new subtype tree:

\begin{lstlisting}[caption=Versioning Using Subtyping]
//file appv1_1.module
module appv1_1;

//file appv1_2.module
//version 1.2 is backward compatible to 1.1
module appv1_2 extends appv1_1;

//file appv2_0.module
//but version 2 is no longer backward compatible
module appv2_0; 
\end{lstlisting}

In this manner backward compatibility can now be expressed in 
and checked by the type system, instead of just being written
down in separate documentation.

Overrides can be used in a similar manner. A module can be declared to only 
override the previous versions to which it is backward compatible.

In addition, extends allows patch releases that do not contain 
the entire previous version's classes, as already demonstrated in
the previous section. A module system may also internally generate
a subtype module to add to an existing module's classes.

Interfaces can also be used to model version ranges, and even feature
compatibility across versions. Versions that are in the same version 
range or are feature compatible implement the same interface.

\begin{lstlisting}[caption=Interfaces as Properties]
//file appv1.module
module appv1 implements 
	appv1to2, 
	supportsfeature1;

//file appv2.module
module appv2 implements 
	appv1to2, appv2to4, 
	supportsfeature1;

//file appv3.module
module appv3 implements 
	appv2to4, 
	supportsfeature1, 
	supportsfeature2; //supports new feature

//file appv4.module
module appv4 implements 
	appv2to4, 
	supportsfeature2; //no longer supports feature1
\end{lstlisting}

In general, interfaces can be used as a tag for any arbitrary constraint, 
in a similar manner that empty interfaces are used in Java to mark classes 
that satisfy a certain property.

It should be noted that extends, implements and overrides are all orthogonal,
so a module can be a subtype of another module, override a set of modules and
implement a set of interfaces. This allows a module author to reuse portions
of the supertype module, specify certain properties that the module satisfies
using module interfaces, and specify a set of modules it overrides (which may
include its supertypes, even though a subtype already implicitly overrides
its supertype).

These extends, implements and overrides declarations do not have to be explicit.
Common usage of existing module systems mostly already assume a that a 
module is a subtype of another module of the same name if its version 
number is higher. Existing module systems can generate these version-related types 
internally, while still allowing explicit declarations for patch releases 
using extends and using module interfaces for expressing non-version
related constraints.

As shown on the previous section, merge and replace can be used to 
update references to old versions of a module with a newer version.
In addition, it is now possible to make sure that modules use
the exact same version of the module they import.

\SubSection{Imports and Exports}

Imports and exports are directly implemented by module imports and export package.
OSGi's loose coupling using export and import packages can be simulated by
the module system internally generating a weak interface that satisfies the import
package specification, finding a visible module that matches that weak interface, and
inserting an import and replace module declaration in the module that contained the import package
declaration.

Module interfaces allow for looser module coupling, allowing for indirect 
references to modules depending on their export package signature. The module private
modifier allows for tighter information hiding.

Packages, however, are still just class modifiers. A possibly useful addition to 
\texttt{package-info.java} is the specification of required classes as annotations:

\begin{lstlisting}
//file package-info.java
@RequiredClass("Parser")
@RequiredClass("Scanner")
package org.x.y.lib;
\end{lstlisting}

The annotations would cause an error in a build if one of the required classes were not
found in the package. This would reduce the problems caused by split packages, and it should
also be possible to use these in a manner similar to module interfaces to specify the kind
of package that an application requires, without cluttering the already crowded module
file specifications of current systems. 

\SubSection{Instantiation}

Import own is derived directly from iJAM, and is improved with an explicit exporting 
mechanism and a module lookup system using module qualifiers. Singleton module instances mirror
OSGi singletons. The module qualifier in type references and
imports allow fine-grained selection of the class to load in contrast
to fixed loading strategies. 