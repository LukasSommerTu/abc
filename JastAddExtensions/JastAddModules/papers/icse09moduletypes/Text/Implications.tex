Treating modules as types provide a more elegant solution to JAR hell, as it removes the dependence
on classpath order and allows disambiguation of multiple visible types with
module qualifiers. In addition, we have shown that the type system can express
constraints such as backward compatibility that were previously only documented 
and not checked at build or load time.

Note that the syntax only represents the type system, it is not the only way
to implement it. As mentioned, subtyping is already implicitly implemented
by versions. Other existing features such as loose coupling in OSGi can
also be modeled, i.e. by using weak interfaces and a generated import and replace
declaration. The type system is also not limited to
the current implementation. For example, the import declaration can be 
extended to use OSGi's extensive constraint features for packages and bundles.

Conflicts from possibly split packages can now be handled as an ambiguous type error 
instead of nondeterministically choosing the class to load.

Separate compilation of modules with module interface dependencies would require
the ability to compile and lookup module qualified names during runtime. We believe
it is possible to do so with a classloader implementation of the module system, and
this is left as future work.

The merge operator at present is relatively simple, it just picks the most specific module
among the merge targets. A natural extension of this is for a merge to create a 
synthesized module that merges the contents of its merge targets, assuming there are no
conflicts. Conflict analysis and resolution as well as the relation of the synthesized module
to existing modules is left as future work.

Merge and replace are currently just used in the module's ``constructor'', that is, in
the module's instantiation instructions. These may be extended to be executed in the context
of a running repository, and may provide the basis for a language for module upgrades and replacement.

Merge and replace are imperative operations on modules. As the module system
is now not entirely declarative, a method of viewing the results of these operations
for debugging becomes desirable, perhaps even necessary.

Collections of classes are now typed. If aspects in aspect-oriented programming are viewed as program transformations,
modules can be used as types to limit the kinds of programs that an aspect may modify, even though 
this contradicts the standard view that aspects are not program transformations. 
Aspects are already dependent on the structure of the programs they affect (i.e. 
an advice or intertype declaration that introduces code to type C most likely
depends on the existence of C), and typing will allow checks on an aspect's applicability
and extent.
This may lead to a saner use of aspects that will preserve the modularity properties
that have made object-orientation so well-accepted.

