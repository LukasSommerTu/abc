%------------------------------------------------------------------------- 

As the previous section shows, while the problem of java modules have been
well studied and multiple solutions exist for different aspects of the problem,
there are still remaining issues such as ambiguous lookups, split packages and
versioning constraints that remain. We propose that these can be solved by
defining a type system on modules, which will model the current usage
of modules while allowing for the expression of constraints and relations
that were previously not possible.

We now non-formally define a type system over collections of classes. 
A module is a collection of classes, and we consider it to be the type of 
the collection of classes that belong to it. We then proceed to define
relations and operations on modules, and in a later section, show how
these model current module usage.

\SubSection{Declaration and Membership}

Modules are defined in a {\tt .module} file that begins with a module
declaration:

\begin{lstlisting}
//File org.x.y.mymodule.module
module org.x.y.mymodule;
\end{lstlisting}

Module membership is declared using syntax that was proposed 
for JSR 294 \cite{superpackageChange}. Membership of a compilation unit is 
declared by adding a module declaration. If a package declaration is present, 
the package name is used as the package name of the file within the module.

\begin{lstlisting}
//File C.java
module org.x.y.mymodule;
package org.x.y;
public class C { ... }
\end{lstlisting}

Split packages (packages whose types belong to more than one module) cause hard to 
find problems in current module systems\cite{iJAMComments}. To avoid this, 
a way to specify package membership in a module allowed by declaring the module membership
in the {\tt package-info.java} file of that package.

\begin{lstlisting}
//File package-info.java
module org.x.y.mymodule;
package org.x.y;
\end{lstlisting}

If the module membership specified in a compilation unit's {\tt package-info.java}
file is in conflict with that specified in the file itself, the membership
declared in the file takes precedence, but a warning is issued to flag a possibly
split package.

%------------------------------------------------------------------------- 
\SubSection{Imports}

For a module to be able to access any of another module's classes,
it must first import an instance of that module. This way, imports form
a hard constraint on the visibility of classes.

There are two ways to import a module. The first is to import the singleton
instance of the module. This is done in the following manner:

\begin{lstlisting}
//imports the singleton instance
import org.x.y.othermodule;
\end{lstlisting}

Importing the singleton instance of a module allows you to share the classes
of that module with other modules that also import it.

Similar to iJAM \cite{iJAM}, a module may also import its \textbf{own} instance of
a module. It is also allowed to rename that instance to allow for multiple
instances of the same module to exist in the same context.

\begin{lstlisting}
//imports an own instance
import own org.x.y.othermodule;
//imports on own instance with an alias
import own org.x.y.othermodule as myothermodule;
\end{lstlisting}

Unlike iJAM, renaming an instance does not implicitly allow other
modules to access that instance. To allow access to other modules, the import
must be exported:

\begin{lstlisting}
//imports an own instance, and makes it available to
//other modules
import own org.x.y.othermodule export as othermodule;
\end{lstlisting}

The visibility of the module instances become important for module
qualified name lookups and the merge and replace operations described in later
sections.

\SubSection{Exported packages and the Module Modifier}

We allow exported packages similar to {\tt export-package} in OSGi bundles. Unlike OSGi, however,
exporting a package does not automatically allow other modules to gain
access to these packages. A module A must first import another module B
before gaining access to B's exported packages.

Exported packages are declared in the {\tt .module} file:

\begin{lstlisting}
module org.x.y.mymodule;
//allow other modules access to these packages
export package org.x.y.parser, org.x.y.lexer;
\end{lstlisting}

You may also export all the packages that belong to a module by using
the * wildcard. This is the only way to export classes that belong to
the default package in a module.

\begin{lstlisting}
module org.x.y.mymodule;
//export all packages
export package *;
\end{lstlisting}

We also allow the \textbf{module} access modifier proposed for JSR294 \cite{JSR294}
which applies to classes, fields and methods. Module private access only allows
accesses from classes that belong to the same module.

\begin{lstlisting}
module class C {
	public void publicMethod() {...};
	module void moduleMethod() {...};
}
\end{lstlisting}

\SubSection{Module Qualified Type References}

Type references can now be qualified with module names to resolve ambiguous accesses.
In this example, we have a module {\tt myapplication} importing 
two instances of module {\tt xmlparser},
which contains the class {\tt xmlparser.Parser}. 
A class {\tt }C in {\tt myapplication} can then choose from which 
instance of {\tt xmlparser} to load the class:

\begin{lstlisting}[caption=Module Qualified Type References,numbers=left]
//File myapplication.module
module myapplication;
//import two instances of xmlparser
import own xmlparser as parser1; 
import own xmlparser as parser2;

//File xmlparser.module
module xmlparser;
export package xmlparser;

//File XMLParser.java
//Classes in this file belong to module .xmlparser
module xmlparser; 
package xmlparser;
public class XMLParser{...}

//File C.java
//Classes in this file belong to module .myapplication
module myapplication; 
public class C{
	//This resolves to the class in the first instance
	parser1::xmlparser.XMLParser p;
	
	//This resolves to the class in the second instance
	parser2::xmlparser.XMLParser p2;
		
	//ambiguous type error 
	xmlparser.Parser p3 = xmlparser.Parser(); //error
}
\end{lstlisting}

Module qualified names are also supported in import declarations,
and can reference exported modules from indirectly imported modules.

\begin{lstlisting}[caption=Module Qualified Imports,numbers=left]
//file org.x.y.parserapplication.module
module org.x.y.parserapplication;
import own parser;

//file org.x.y.parser.module
module org.x.y.parser;
import own org.x.y.scanner export as scanner;
export package org.x.y.parser;

//file org.x.y.scanner.module
module org.x.y.scanner;
export package org.x.y.scanner;

//file Parser.java
module parser;
package org.x.y.parser;
public class Parser{...}

//file Scanner.java
module scanner;
package org.x.y.scanner;
public class Scanner{...}

//file ParserApplication.java
module parserapplication;
package org.x.y.parserapplication;

//single type import
import parser::org.x.y.parser.Parser;
//on demand import
import parser::scanner::org.x.y.scanner.*;

public class ParserApplication {
	//uses the Parser class from module parser
	Parser p = new Parser(); 
	//uses the Scanner class from module scanner
	Scanner s = new Scanner(); 
	...
}

\end{lstlisting}

Types and packages from directly imported modules do not need to
be module qualified. In the example above, one could have used
\begin{lstlisting}
import org.x.y.parser.Parser;
\end{lstlisting}
on line 29.

Types that do not belong to a module are considered to be members of
the ``default'' module. This module has a blank name, and its singleton
instance is implicitly imported by all modules. So, a module-less
type {\tt javax.swing.Action} can be accessed in this way:

\begin{lstlisting}[caption=Default Module Lookups]
module somemodule;
//import from the default module
import ::javax.swing.Action;
class MyClass {
	Action a = new Action();
}
\end{lstlisting}

It is not strictly necessary to always module qualify accesses to types in
the default module, as long as there are no packages in the current module
or its imported modules that have the same package and type name. However,
if these do exist, it is necessary to use the module qualified name to access
these classes to avoid an ambiguous type error.

%------------------------------------------------------------------------ 
\SubSection{Subtyping}

We now define a subtyping relation on the module type. A module is declared
to be the subtype of another module by using the \textbf{extends} keyword:

\begin{lstlisting}
module org.x.y.parserV1_2 
	extends org.x.y.parserV1_1;
\end{lstlisting}

A subtype module inherits all the import declarations and export package declarations 
of its parent module, and can add some more of its own. It also inherits the member 
compilation units (and the types therein) of its supertype.

If a subtype module contains a type with the identical package and type name
to another type which is a member of its supertype module, 
this shadows the type in the supertype module for all type references that
originate from the subtype module, and any other modules that import an
instance of the subtype module.

The following example demonstrates module subtyping and shadowing. The module
{\tt parserv1\_1} contains the type {\tt parser.Parser}, and imports
the module {\tt scanner} which contains {\tt scanner.Scanner}. The module
{\tt parserv1\_2} then extends {\tt parserv1\_1}, while having its
own version of {\tt parser.Parser} and adding a new class {\tt parser.XMLParser}.

\begin{lstlisting}[caption=Module Subtyping]
//file parserV1_1.module
module parserV1_1;
import own scanner export as scanner;
//say there is a class Parser in this package
export package parser;

//file parserV1_2.module
module parserV1_2;
//say that ParserV1_2 has its own version of 
//parser.Parser, and an additional class
//parser.XMLParser

//file scanner.module
module scanner;
//say there is a class Scanner in this package
export package scanner;

//file MainV11.java
module parserV1_1;
import parser.Parser;
import scanner.Scanner;
public class MainV11 {
	//this references the v1_1 parser
	Parser p = new Parser(); 
	Scanner s = new Scanner();
	...
}

//file MainV12.java
module parserV1_2;
import parser.Parser;
import scanner.Scanner;
public class MainV12 {
	//this references the v1_2 parser
	Parser p = new Parser(); 
	//this references the inherited Scanner class
	Scanner s = new Scanner();
	...
}

//file multiversion.module
module multiversion;
import own parserV1_1 as parserV1_1;
import own parserV1_2 as parserV1_2;

//file MultiversionMain.java
module multiversion;
public class Main {
	//this references the v1_1 parser
	parserV1_1::parser.Parser p1 = 
		new parserV1_1::parser.Parser();
	//this references the v1_2 parser
	parserV1_2::parser.Parser p2 =
		new parserV1_2::parser.Parser();
}
\end{lstlisting}

%------------------------------------------------------------------------- 
\SubSection{Merge}

It is often necessary to ensure that the exported modules of a set of modules
that you import point to the same module instance, to ensure that these
modules are able to share types. As an example, say you have two database
backend modules {\tt mysqlbackend} and {\tt postgresbackend} with a method
{\tt ResultSet runQuery()}. Further suppose that {\tt ResultSet} is
defined in another module {\tt sqltypes}, which is \textbf{own} imported by both backends.
In order for the return types of both backends to be type compatible, the
reference to {\tt sqltypes} in both backends must also point to the same
instance.

To ensure that the {\tt sqltypes} instances in both {\tt mysqlbackend}
and {\tt postgresbackend} point to the same instance, we use the 
\textbf{merge} operation. Merge operations take a set of merge targets,
and an alias to refer to the merged module:

\begin{lstlisting}
merge m1,m2,m3 [export] as alias;
\end{lstlisting}

The merge operation requires that the types
of the modules being merged be the same or that one is an ancestor of
the other in the subtyping relation (i.e. they are all on the same subtype path). 
It then creates a new instance
of the most general type that is still compatible with the types of the
modules being merged, and then points those module references to the new 
merged module. The old module instances are discarded, unless there
are other references to them that are not part of the merge.

\begin{lstlisting}[caption=Merge]
//file mysqlbackend.module
module mysqlbackend;
import own sqltypes export as sqltypes;
//say query has a class MySQLQuery
export package query;

//file postgresbackend.module
module postgresbackend;
import own sqltypes export as sqltypes;
//say query has a class PostgresQuery
export package query;

//file sqltypes.module
module sqltypes;
//say that sqltypes contains types.ResultSet
export package types;

//file myapplication.module
module myapplication;
import own mysqlbackend;
import own postgresbackend;
//merge the sqltypes
merge 
	mysqlbackend.sqltypes, 
	postgresbackend.sqltypes 
export as sqltypes;

//file Main.java
module myapplication;
import mysqlbackend::query.MySQLQuery;
import postgresbackend::query.PostgresQuery;
import types.ResultSet;
public class Main {
	void someMethod(){
		ResultSet s = null;
		//this would not be possible if the sqltypes
		//module instances are different
		if (mysql) {
			s = new MySQLQuery(...).query();
		} else if (postgres) {
			s = new PostgresQuery(...).query();
		}
		...
	}
}
\end{lstlisting}


It must also be noted that the merged instance is a direct import of
the module where the merge was declared and, as the example above shows,
can be re-exported. This means that types in the merged module {\tt sqltypes} 
can be accessed in {\tt myapplication} without the need to use module 
qualifiers unless the type reference is ambiguous. As the example above
shows, {\tt ResultSet} was imported without a module qualifier in
{\tt Main.java}.

Merges are also implicitly inherited by subtype modules. The merge sequence for
a module is given by the list of merge sequences of a module's supertypes, starting
from the farthest ancestor. There is no way to exclude a supertype's merges, as
this may reduce the module signature (the set of module references available) of
the supertype module, which could lead to both internal and external clients of
the module breaking.

\SubSection{Overrides}

Subtyping allows extension or patching of a module without actually
rebuilding a completely new module. However, it does have the disadvantage
of being dependent on the existence of its supertype modules. To get
around this limitation, we define the \textbf{overrides} relation to allow
a module to completely replace another module. A module is declared
to override another using the \textbf{overrides} keyword:

\begin{lstlisting}
module parserV2 
	overrides parserV1_1, parserV1_2;
\end{lstlisting}

Unlike subtyping, an overriding module does not inherit anything from the
modules it overrides. However, it \textit{must} provide the same exported modules
and packages as its overridden modules to satisfy the external 
clients of these modules.

As is expected, override is inherited by subtype modules. This allows a subtype
of an overriding module to override the same modules as its supertype.

Declaring an overriding module does not automatically change over all references
to the overridden module to use the overriding module. This is done using the
\textbf{replace} operation discussed below.

\SubSection{Replace}

While merges can be used to re-bind module references, they are only applicable 
for modules that are subtypes of each other. As overriding modules are not
necessarily subtypes of the modules they override, we introduce the \textbf{replace} operation
to substitute overriding modules for overridden modules. The replace operation
takes a set of target module references to rebind, and the module reference pointing to the 
instance to which the targets are to be bound:

\begin{lstlisting}
replace m1,m2,m3 with m4;
\end{lstlisting}

The replace operation
changes the binding of module references to point to an existing
module instance. A module may replace another if its type is the same or a subtype
of that module, or if its type overrides that of the other.

Continuing the parser example given in the subtyping section above, say we have two 
modules, {\tt staticanalyzer} and {\tt javadocgenerator}, each using {\tt parserv1\_1}
and {\tt parserv1\_2} respectively. We now wish to create a new module that uses these, 
but updates them to {\tt parserv2}, which overrides both {\tt parserv1\_1} and {\tt parserv1\_2}.

\begin{lstlisting}[caption=Module Replace]
//file staticanalyzer.module
module staticanalyzer;
import own parserv1_1 export as parser;
...

//file javadocgenerator.module
module javadocgenerator;
import own parserv1_2 export as parser;
...

//file parserv2.module
module parserv2 
	overrides parserv1_1, parserv1_2;
...

//file myapplication.module
module myapplication;
import own staticanalyzer;
import own javadocgenerator;
import own parserv2;

replace 
	staticanalyzer::parser,
	javadocgenerator::parser
	with parserv2;
\end{lstlisting}

Note that even though the modules {\tt parserv1\_1} and {\tt parserv1\_2}
are not present in the build, it is able to compile successfully when the
module references to these missing modules are replaced with the overriding
module {\tt parserv2}.

As with merges, replaces are implicitly inherited by subtype modules.

%Merge and replace make different modules sure to use the same
%set of classes (avoids nondeterministic class loading). Also, can
%be used to update modules to use newer versions of their imported
%modules. They can also be used to access the namespace of indirect imports
%without qualifiers by merging the indirect imports with a directly
%imported instance of that module.

\SubSection{Module Interfaces}

Extending the object-oriented metaphor, with module types being classes, we 
now define \textit{module interfaces}, which act similarly to interfaces in Java. A module
interface contains no imports, replaces, merges or even member classes, but they
are allowed to have a set of export package declarations. A module implementing
a module interface must contain and export the packages specified in the interface.

\begin{lstlisting}[caption=Module Interfaces]
//file org.x.y.math.module
module org.x.y.math implements 
	org.x.y.calculus,
	org.x.y.matrices;
export package 
	org.x.y.math.calculus,
	org.x.y.math.matrices,
	org.x.y.math.complex;
	
//file org.x.y.calculus.module
module_interface org.x.y.calculus;
export package org.x.y.math.calculus;

//file org.x.y.matrices.module
module_interface org.x.y.matrices;
export package org.x.y.math.matrices;

//file engine3d.module
module engine3d;
import own org.x.y.matrices export as matrices;

//file enginephysics.module
module enginephysics;
import own org.x.y.calculus export as calculus;

//file myapplication.module
module myapplication;
import own engine3d;
import own enginephysics;
import own org.x.y.math;

replace 
	engine3d::matrices, enginephysics::calculus
	with org.x.y.math;
\end{lstlisting}

Rules for module interface subtyping are similar to those of interfaces in Java:
interfaces can only extend interfaces, and they can not implement other
interfaces. 

Since interfaces do not actually contain any classes, they must be replaced
by a non-interface module when compiling a fully working system. This would
have to be relaxed for separate compilation of modules that reference interfaces
to be possible, and this is discussed a bit more in the section on future work.

\SubSection{Weak Module Interfaces}

It may be the case that an application developer knows the packages that
he wishes to use, but the provider of the module that contains these packages 
did not define a module interface that specifically contains those packages. For
this case, we define a \textit{weak module interface}. Weak module interfaces act similarly
to normal module interfaces, except that it is also implicitly implemented
by all modules that satisfy its export package signature \cite{mcdirmid01jiazzi, componentnextgen}, 
even if these modules did not explicitly declare that they implemented the interface.

Continuing the example given above, weak interfaces allow us to define 
our own interfaces to a module without the module provider's knowledge:

\begin{lstlisting}[caption=Weak Interfaces]
//file mymathinterface.module
weak_module_interface mymathinterface;
export package 
	org.x.y.math.calculus, org.x.y.math.complex;

//file myfourier.module
module myfourier;
import own mymathinterface export as math;

//file myapplication.module
module myapplication;
import own myfourier;
import own org.x.y.math;
//replace the interface with an implementing module
replace myfourier::math with org.x.y.math;
\end{lstlisting}

\SubSection{Type Lookup Sequence}

This module system, as with any module system for Java, changes the way that type
references are looked up. The following pseudocode shows how type lookup is done in
the type system defined above. 

\begin{lstlisting}[caption = Type Lookup, tabsize=2, morekeywords={method}]
//lookup for unqualified names in a CU
method CU.lookup(typeName) {
	//get the module of which the CU is a member
	Module = CU.getParentModule();
	if (Module != null) {
		lookup classes in CU
		lookup classes in single type imports
		//lookup in module and its supertypes only
		Module.lookup(null, CU.package(), typeName, false)
		lookup classes in on-demand imports
		//lookup in module, including direct imports
		Module.lookup(null, CU.package(), typeName, true)
		lookup primitive types
		lookup automatic imports (java.lang)
	} else {
		normal java lookup
	}
}

//lookup for qualified names
//takes the module qualifier, package
//qualifier and typeName of a type reference
//and a boolean value lookInImports
method Module.lookup(moduleName, packageName, 
							typeName, lookInImports) {
	if (special packageName (java.lang)) {
		lookup in defaultmodule
	}
	if (moduleName == null) {
		lookup in thismodule
		lookup in each successive supertype
		if (lookInImports) {
			lookup in direct imports
		}
	} 
	else {
	//lookup through the module qualifier
	contextModule = lookupModule(moduleName);
	contextModule.lookup(``'', packageName, 
												typeName, false);
	}
}
\end{lstlisting}

The lookup rules follow the Java way of looking up types,
starting from the most local proceeding to the most global. Types are looked up
first in the same compilation unit, then in the single type imports,
then in the types that belong to the same package in the module and its supertype 
modules, then the on-demand imports, then the member types of the same package
in directly imported modules, and finally to the primitive types and the 
implicit {\tt java.lang} imports. It was a conscious decision to make on-demand imports
come first before the lookups to directly imported modules. This is because the on-demand import
is closer to the type reference being resolved, being in the same source file instead of
on a separate module specification file.

\SubSection{Constraints on Modules}

Cycles involving \texttt{import own} are not allowed, as this leads to an infinite loop of
module instance creation. However, cycles of singleton imports are allowed.

Subtyping is single inheritance, as in Java. This avoids ambiguous replace and
merge inheritance order in a subtype module.

Merges cannot be done on singleton instances or modules accessed through a
singleton instance. This is to make sure that other clients using the singleton
instance are not affected by the change.

As with merges, replace targets can not be a singleton instance or a module
accessed through a singleton instance, for similar reasons. 

Merges can only be done on submodules in the same path to the root module in
the submodule tree. Furthermore, this should not change the module export signature
of the module being submoduled. This is to guarantee that both internal and external
clients of the module can still rely on their module qualified lookups.

Exported packages in non-interface modules imply that there are classes that actually 
belong to that package. Otherwise module interfaces become less useful.

As mentioned, implementing a module interface means exporting the packages the interface
exports. This is to ensure that the signature contract with the clients of
the interface are satisfied.

All references to an interface module must be replaced. Otherwise all lookups through 
that reference will fail as the interface is empty. 