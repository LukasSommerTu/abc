This provides a description of the JastAdd module system.

\subsection{Module System Overview}

A module consists of aspects and classes, and defines the set of external
modules that are visible to the module's members. The module system is based
on the object-oriented java module system presented in \cite{modulesastypes}

\subsection{Declaration, Membership and Exports}

A module is defined in a \texttt{.module} file, and is headed by the module's name:

\begin{lstlisting}[caption={Module Declaration}]
//file prettyprinter.module
module prettyprinter;
...
\end{lstlisting}

Membership to a module is defined in the compilation unit of the members,
similar to a package declaration:

\begin{lstlisting}[caption={Module Membership}]
//file prettyprinter.jrag
module prettyprinter; //module membership
public aspect PrettyPrinter {
	public abstract String Expr.prettyPrint();
	
	public String Add.prettyPrint() {
		return getLeft().prettyPrint() + 
				"+" + getRight().prettyPrint();
	}
	
	public String IntLit.prettyPrint() {
		return getIntLit().toString();
	}
}

//file Expr.java
module asttypes; //module membership
package expr; //package declaration
public class Expr {
...
}

\end{lstlisting}

Module and package declarations can coexist in a single compilation unit.
Modules can contain aspects and classes that span multiple packages.

Packages are also not implicitly visible outside the module unless an
\texttt{export} declaration is provided for that package. These declarations
are placed in \texttt{.module} files.

\begin{lstlisting}[caption={Export Package}]
//file asttypes.module
module asttypes;
export package expr, stmt; //export expr and stmt

//file prettyprinter.module
module prettyprinter;
export package *; //export all packages
\end{lstlisting}

As the example shows, an export package may contain a list of packages, or the
wildcard \texttt{*}, which exposes all packages in the module. Any types that
belong to a package that is not exported are not visible from outside the module.

\subsection{Imports and Instantiation}

Module definitions also contain import declarations, which specify which
other modules are visible to 

\subsection{Merge}



\subsection{Extension}



Imports and instantiation/merge, extension, access control.

\subsection{ITD Calculator with Modules}

\subsection{Evaluation}

