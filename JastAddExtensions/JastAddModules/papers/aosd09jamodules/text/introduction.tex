Inter-type declarations (ITDs) provide a powerful yet simple modularisation
mechanism. The possibility to extend existing classes modularly without
ahead of time planning is not only useful to separate different concerns
but also extremely convenient for modular extensibility when software
evolves.

A common critisizm of ITDs is their global scope which arguably leads to
poor information hiding, a topic that has gained renewed interest with the 
emerging support for modules in Java 7. Another disadvantage is that the 
class hierarchy is destructively updated, preventing multiple variants of 
classes with different sets of ITDs applied. It is worth noting that 
this drawback is not shared by more traditional extension mechanisms such 
as visitors.

These problems exists to a certain extent in plain Java programs as well
and there has been a wealth of recent work on module systems to improve on
status quo. 

The emerging support for modules in Java 7 enhances information hiding and
extended module proposals such as Strnisa gives hope for simulatneous
deployment of multiple versions of the same library in different modules.

Modules provide information hiding at a level higher than packages. Module
systems like OSGi bundles\cite{OSGi4} and the proposed Java module system\cite{JSR277}
allow the explicit definition of the dependencies of a module. We can extend
the information hiding features of these module systems to extend to aspects, 
so as to limit the scope of ITDs.

Another module system, iJAM \cite{iJAM}, adds explicit module instantiation, 
which allows multiple versions of the same module to coexist in a single compilation.
This is built upon by some of our previous work \cite{modulesastypes}, which 
adds the idea of treating modules as object-oriented types, with instantiation
similar to iJAM and additional operations to allow explicit and fine-grained
module instance sharing. 

ITDs are global in their nature which makes local
reasoning somewhat difficult, and as an extensibility mechanism they can be
improved by enabling deployment of multiple versions of libraries, 
each woven a with different set of ITDs.


Previous work show how aspects can be improved using modules for point-cut
and advice.
Aspects don't work very well without modules, due to global scope, and
implicit dependencies.
In this paper we present a module system that supports inter-type
declarations and improve their use when extending a system in a modular
fashion.

We believe that such benefits are even more important for a system with
inter-type declarations. 


%Aspect instantiation has always been a sticky subject (e.g. doubly applied
%pointcuts in AspectJ abstract aspects)


In our previous work we explain how to use ITDs as one of the main
modularisation mechanisms when building extensible compilers. Our current
work involves using the same techniques to generate IDEs for a wide range
of dialects of Java. In such an IDE we may for instance want to use numerous variants of the
same frontend that slightly differ to support different dialects. We also
want to use a pure frontend for error checking while a backend is also
needed to support code generation.
That work highlights some defiencies to ITDs from an extensibility point of
view compared to the more traditional use of visitors.

Traditionally, this was done using visitors. Inter-type declarations have the advantages of
not requiring too much ahead of time planning, the ability to add state, minimal boiler-plate code, 
and being less error prone since you don't have to adhere to framework conventions to enable dispatch.

However, ITDs do not provide the same level of extensibility as visitors.
ITDs are currently destructive updates of the class hierarchy. The base version and the
extended version can not co-exist. This is the main motivation for our
work. 

Based on previous work on object-oriented modules for Java \cite{modulesastypes}, 
we have implemented the proposed module system as an extension to the
Jast\-Add Extensible Java Compiler which contains support for amongst others
ITDs. 
To evaluate the module system we performed a case study where the above
mentioned extensible Java compiler built using ITDs was refactored to use
the proposed module system.
The proposed module system solves some of these problems completely in an
elegant fashion, while the burden of other problems are slightly lowered.

These are the main contributions of this paper:
\begin{itemize}
\item The design of a module system for ITDs that improves information
hiding and extensibility.
\item An implementation as a modular extension to the Jast\-Add Extensible
Java compiler.
\item A case study where extensible Java compiler is retrofitted to use the
proposed module system.
\end{itemize}

The rest of this paper is structured as follows: In
Section~\ref{section:itdvisitors} we present a detailed example that
highlight the merits and deficiencies of ITDs compared to a visitor based
approach. A module system that shows how that example can be improved is
presented in Section~\ref{section:jastaddmodules} and we present a case
study where an extensible Java compiler is retrofitted to use that module
system in Section~\ref{section:casestudy} where we also discuss 
the advantages it brings compared to the original impelementation. A
brief overview of the module system implementation is presented in
Section~\ref{section:implementation} and we discuss related work in
Section~\ref{section:related} and conclude in
Section~\ref{section:conclusions}.

