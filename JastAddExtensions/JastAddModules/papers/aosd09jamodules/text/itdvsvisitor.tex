In this section we present two alternative implementations of a small
calculator, one using visitors and one using ITDs. It will be used as a
running example throughout the paper to illustrate the advantages and
disadvantages of the respective solution. In Section~\ref{jastaddmodules}
we then present a module system for ITDs that removes some of the presented
deficiencies.

\subsection{Running Example: Calculator}
Consider a tiny calculator supporting integer literals and addition, thus 
supporting composite expressions like: 1 + 2. It may seem trivial but its 
implementation illustrates some important exensibility challenges that we 
address in this paper. The calculator supports various computations on top 
of this structure, e.g., evaluating its value and printing its string
representation.

To evaluate the extensibility of a visitor based solution and an ITD based
solution we extend the calculator with two small extensions: first support
for subtraction and then support for an alternative string representation 
where constants are printed using natural language, i.e., one, two, 
three, etc.

TODO: %Introduce the Expression problem.

Notice that this is a simplified example of an compiler. Information is
for instance only propagated bottom up so there is no need for parameter
propagation during a visit. Parameters would require a significantly more 
complex visitor due to the requirement on contravariance in the argument 
for type safety would require much more complex genericity. We chose an 
example that is flattering for visitors to be fair when arguing that ITDs 
are a more suitable solution even in this case. While the example is 
tiny the same structure is commonly used in compilers. In~\cite{} we argue 
the merits of using ITDs and attribute grammars rather than visitors in 
extensible compiler construction, using a complete Java compiler as a 
major case study.

\subsection{Visitor Implementation}
A visitor based solution uses the composite design pattern to model the
static structure of expressions: one class for each language element and
composition of language elements forming a tree structure. I.e., the 
Add node has references to its left and right child respectively, while 
the Int node has a single filed holding its value.

Each node type also contains an accept method that performs an additional
dispatch to select the code in the visitor for the concrete node type. 
This double dispatch scheme is the key to dispatching both on the node 
type and the visitor. We can thus select a particular method that is
selected based on both the dynamic type of the recieving object and the 
dynamic type of the visitor, provided as a parameter. 

A visitor is then simply a class that holds an implementation for each concrete 
class type that may accept the visitor. The Eval visitor computes the 
value for Int and Add, while the Print visitor computes a string 
represenation for each expression in a similar fashion. The visitor code 
is required to visit children indirectly using the accept method which 
clutters the implementation and is somewhat error prone.

\begin{lstlisting}[caption={Visitor Base}]
abstract class Expr {
}
class Int extends Expr {
  int value;
  <T> T accept(Visitor<T> v) {
    return v.visitInt(this);
  }
}
class Add extends Expr {
  Expr left;
  Expr right;
  <T> accept(Visitor<T> v) {
    return v.visitAdd(this);
  }
}
class Visitor<T> {
  <T> visitInt(Int i) { return null; }
  <T> visitAdd(Add a) { return null; }
}

class Eval extends Visitor<Integer> {
  Integer visitInt(Int i) {
    return i.value;
  }
  Integer visitAdd(Add a) {
    return a.left.accept(this) + a.right.accept(this);
  }
}
class Print extends Visitor<String> {
  String visitInt(Int i) {
    return Integer.toString(i.value);
  }
  String visitAdd(Add a) {
    return a.left.accept(this) + " + " + a.right.accept(this);
  }
}
\end{lstlisting}

\subsection{ITD Implementation}
The ITD implementation of the calculator has a similar implementation of the
node types as the visitor based solution, but the accept methods are no
longer necessary. Instead the aspect Eval introduces an eval method in each
node type. These methods can then be called from other introduced methods
directly without the need for using accept methods to enable dispatch.

\begin{lstlisting}[caption={ITD Base}]
abstract class Expr {
}
class Int extends Expr {
  int value;
}
class Add extends Expr {
  Expr left;
  Expr right;
}
aspect Eval {
  int Int.eval() {
    return value;
  }
  int Add.eval() {
    return left.eval() + right.eval();
  }
}
aspect Print {
  String Int.print() {
    return Integer.toString(value);
  }
  String Add.print() {
    return left.print() + " + " + right.print();
  }
}
\end{lstlisting}

\subsection{Extension}
As an example of extension we first add a subtraction node and enable the
existing analyses to support that new operation. Then, we add a new
analysis that computes a string representation of an expression where the
literals are printed using natural language.

\subsubsection{Visitor extension}
We need to add a new node type Sub that includes the boiler plate code to
support the visitor pattern. Then the base visitor is extended to support
the new node type. This change is not completely modular but a pragmatic
choice to keep the solution simple. We could have used various tricks with advanced
generics concepts, but that would make the solution even more complex and 
clutter the solution with framework code. It should be admitted that the 
current solution is not
completely safe since the default implementation for visitSub is not
suitable for the Eval and Print visitors. 

We also provide extended vistitors
,ExtEval and ExtPrint, to support the extended language.
They can extend the base behaviour and simply provide the increment
needed to support the subtraction operation. The extended version of the
printer can then be refined to display integer constants using natural
language in the separate visitor WordPrint. Notice that functionality from
the base visitor can be reused here as well and only the functionality 
that is refined need to be provided.

\begin{lstlisting}[caption={Visitor Extension}]
class Sub extends Expr {
  Expr left;
  Expr right;
  T accept(Visitor<T> v) {
    return v.visitSub(this);
  }
}
class Visitor<T> { // replace old impl.
  T visitInt(Int i) { return null; }
  T visitAdd(Add a) { return null; }
  T visitSub(Sub s) { return null; } // new
}
class ExtEval extends Eval {
  Integer visitSub(Sub s) {
    return a.left.accept(this) - a.right.accept(this);
  }
}
class ExtPrint extends Eval {
  String visitSub(Sub s) {
    return a.left.accept(this) + " - " + a.right.accept(this);
  }
}
class WordPrint extends ExtPrint {
  Integer visitInt(Int i) {
    .. convert i to the strings "one", "two", etc.
  }
}
\end{lstlisting}

\subsubsection{ITD extension}
The ITD-based solution simply adds a new node type and introduces methods in that
node type to make the evaluator and printer support subtraction.
The natural language based printer is slightly more complicated. The simple
model of introducing new methods in an existing class is not as convenient
in this case. We can not keep the implementation printing numbers and reuse
the other printing functionality while still changing it to print using
natural language. The word printer therefore choses to take precedence over
the other definitions of visitInt making it the only implementation, thus
disabling the normal printer.

\begin{lstlisting}[caption={ITD Extension}]
class Sub extends Expr {
  Expr left;
  Expr right;
}
aspect ExtEval {
  int Sub.eval() {
    return left.eval() - right.eval();
  }
}
aspect ExtPrint {
  String Sub.print() {
    return left.print() + " - " + right.print(); 
  }
}
aspect WordPrint {
  declare precendence: WordPrint, *;
  Integer Int.visitInt(Int i) {
    .. convert i to the strings "one", "two", etc.
  }
}
\end{lstlisting}

\subsection{Discussion}
While both presented solutions are quite similar there are some important
differences:

Visitors require some ahead of time planning with boiler plate code in the 
node types while the ITD solution can modularly update an existing class 
hierarchy if needed. The actual visitors can be updated in a 
modular fashion using inheritance and overriding but the user of these 
analyses need to update the code to use the extended versions. ITDs can 
be used to directly add support for existing analyses in the new node type 
in a completely modular fashion. 

The visitor based solution need to include framework code to enable the 
double dispatch pattern which clutters the implementation and is somewhat 
error prone. The ITD solution on the other hand is quite 
straightforward with a clean implementation.

The base visitor could not be modularly extended in the presented example.
There are solutions based on advanced usage of generics that support
modular extension but that requires much more framework code and clutters
the solution even more. Moreover, the user need still replace uses of the
old visitors with uses of the new visitor.
(as long as we don't try to run Eval
or Print on a tree containing sub nodes. could probably be fixed using more
complex visitor)

Visitors can be inherited and overridden to refine behaviour in a visitor
being extended. This enables both visitors to be used in the same system
while the ITD solutation can only use either the original implementation 
or the refined implementation. 

Another advantage of the visitor based solution is that each class can be
modularly type-checked and compiled while the ITD based solution requires a
global analysis since methods can be introduced by any aspect in the
system.
