The OSGi Service Platform \cite{OSGi4} is arguably the most developed and
widely used module system for Java at this time. Its main modularity 
construct is the bundle, a specification that applies to the set of class
files that belong to a JAR. Most of the mechanisms
described in the previous sections have been derived from existing usage
of the OSGi platform, and the design of the type system is heavily
influenced by the platform's bundle features. The platform's
capabilities and shortcomings have already been expounded on in the 
previous sections and the operations \textbf{replace} and \textbf{merge},
module extension and module qualified type names add a significant amount of expressibility compared
to OSGi.

JSR 294: Improved Modularity Support in the Java$^{TM}$ Programming Language \cite{JSR294} and
JSR 277: Java$^{TM}$ Module System \cite{JSR277} are two proposals to improve
the Java programming language with modules. JSR 294, now merged with JSR 277, focused
on development modules: structures that allowed for better information hiding within 
a module. It contained the syntax for module membership.
JSR 277 is more focused on deployment modules, and has the same set of design 
concerns as OSGi bundles. JSR 277 modules are very similar to OSGi bundles,
and have a similar set of advantages and disadvantages.

.NET assemblies \cite{netassemblies} are generated modularization structures
for the .NET framework. They handle versioning dependencies and security
requirements, while also allowing for side-by-side execution of multiple
versions of the same assembly.

iJAM \cite{iJAM} has proposed the more flexible \textit{import own}
mechanism for module instantiation and renaming to solve several unintuitive class
resolution results from JSR 294. Sharing of \textit{own} instances was
done through the renaming mechanism. The proposed type system improves
on iJAM by having an explicit instance exporting construct,
allowing for module qualified lookups and having the explicit module operators
merge and replace.

Peter Kriens is a staunch advocate of OSGi and his blog \cite{iJAMComments, superpackagesNoMore}
contains insights on OSGi bundles that have been taken into consideration
in the design of the module type system. His comments on split packages and
the need for a more flexible class lookup mechanism led to the module declarations
in the \textit{package-info.java} and module qualified type names.

The design of various other module systems have inspired several features in the 
proposed type system. MJ modules \cite{corwinMJModules} provide an addition to
the existing Java import declaration to include modules, which has led to module
qualified imports. Typed components \cite{secotypedcomponents} proposed to
view modules as types, but did not have the subtyping and replace operations.
Component Nextgen \cite{componentnextgen} has parameterized interfaces for module imports,
and Jiazzi \cite{mcdirmid01jiazzi} uses class import and export signatures, 
both of which inspired the design of the weak module interface. Smart modules
\cite{Ancona05smartmodules} and Hyper/J \cite{hyperj} proposed the merge operation, 
but in the context of mixin and aspect composition. Module extension and shadowing is inspired by 
virtual classes \cite{virtualclasses89}, which allows the extension of nested classes in a subclass.

Keris \cite{keris05} is a language designed to provide extensibility to components
through the use of virtual classes. It wires a module to its dependencies implicitly,
matching the names of modules available at a certain context to the set of an imported
module's required components. It was not, however, designed to take versions and multiple instances
into account. It assumes that there is only ever one instance
of one module, and is therefore unable to handle an environment where multiple instances and versions
of a module are present. This is also true for the systems mentioned above \cite{corwinMJModules, secotypedcomponents, componentnextgen, mcdirmid01jiazzi, Ancona05smartmodules}, which were designed for composition
and extensibility, but not for versioning.

The implementation of the type system as a module-aware compiler was done in JastAdd \cite{jastadd},
an aspect-oriented compiler construction framework. This has allowed for relatively
rapid development and a simpler pass structure.