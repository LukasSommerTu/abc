This section details the functionality provided by existing module systems that
are in use for Java and other similar languages. 
Several \cite{javajars, OSGi4, netassemblies} have been successfully used in industry 
applications for several years now and should provide a reasonable baseline
for the abilities that the proposed module system should have. This section will also
highlight the shortcomings of existing systems, which will be addressed by the
proposed system.

\subsection{Composition}

Composition is the most basic functionality expected of a module system. There
should be a way for a member of a module to access and be accessed by members of
another module. Ideally, the module system should also have the ability to limit
these accesses to allow for information hiding and loose coupling.

In Java, the classpath provides the most basic level of module composition. 
The classpath provides a list of libraries that are
available during compilation or execution. Classes are looked up in the 
classpath entries in the order they appear, which is fine as long as the sets 
of classes that the entries provide are disjoint. If this is not the case, however, the class may
be loaded from the wrong library. This is JAR hell. Bugs caused by JAR hell
can be difficult to diagnose as the error occurs below the module level, as 
a class cast exception or an invalid method or field access.

Java module systems have more or less been built as solutions to JAR hell. 
OSGi \cite{OSGi4} bundles, JSR 277 modules \cite{JSR277} and .NET assemblies \cite{netassemblies}
have import declarations to define required modules or packages, and use constraints 
to make sure that the modules used by an application satisfy the 
application's requirements. 

However, the current solutions are still not completely satisfactory when there are still multiple
types that satisfy a type reference even with version constraints. In the case
of OSGi, this occurs as split packages \cite{iJAMComments}, when a package is both contained by
two visible modules, modules that are not necessarily versions of each other
and hence not limited by the version import constraints.

\subsection{Versioning}
Adding versions as meta-data to modules is the common solution to multiple
versions of a class. OSGi bundles, the JSR 277 Java Module System
proposal and .NET assemblies have extensive mechanisms for allowing a 
module's client to specify the versions of a module that it supports.

Versioning constraints in OSGi are specified at the point of import. The
listing below shows a sample OSGi bundle \texttt{org.jhotdraw.samples.svg}
that imports the bundles \texttt{org.jhotdraw}, \texttt{net.n3.nanoxml} and
\texttt{org.apache.batik}. It further specifies a version constraint on
the \texttt{batik} import, that its version should be 1.6 (inclusive) as
a minimum, and 1.9 (exclusive) as a maximum.

\begin{lstlisting}[caption=OSGi Bundle Version Constraints]
Bundle-Name: JHotdraw SVG Sample Application
Bundle-SymbolicName: org.jhotdraw.samples.svg
Bundle-Version: 1.0.0
...
Require-Bundle: org.jhotdraw,
 net.n3.nanoxml,
 org.apache.batik;bundle-version="[1.6.0,1.9.0)"
\end{lstlisting}

In .net assemblies, an application depends on the specific version of the 
dependencies it is compiled with, unless it is overriden by a configuration
file. Old versions can be updated to new versions through version redirection,
which is specified in the configuration file. The following example, taken
from the .net MSDN pages, demonstrate version redirection.

\begin{lstlisting}[caption=.net Assemblies Version Redirection]
<configuration>
  <runtime>
    <assemblyBinding xmlns="urn:schemas-microsoft-com:asm.v1">
      <dependentAssembly>
        <assemblyIdentity name="myAssembly" publicKeyToken="32ab4ba45e0a69a1" 
        	culture="en-us" />
        <!-- Redirect references to myAssembly v1.0.0.0 to use v2.0.0.0-->
        <bindingRedirect oldVersion="1.0.0.0" newVersion="2.0.0.0" />
      </dependentAssembly>
    </assemblyBinding>
  </runtime>
</configuration>
\end{lstlisting}

Redirection in .net assemblies may also be specified by the publisher of the
assembly instead of the client. The syntax for the publisher policy is identical
to that above, and becomes the default binding policy for all clients of the assembly.
It can be disabled, however, to allow a client to manually specify the version of
a dependency to use.

However, these constraints are limited to using only the version of the module.
It provides no way to define support for features, for example, where each feature 
will have different backward compatibility specifications across versions. 
Clients will have to manually cross reference the feature documentation with the version releases
to determine which version of a dependency can be used.

\subsection{Extension}

\cite{patchingosgi} patching osgi bundles

\cite{reflexil} \cite{monocecil} .net assembly patching by assembly manipulation.

\subsection{Imports and Exports}

The set of a module's imports and exports specify the packages (or types) that
the module requires and provides, respectively. This affects class
visibility outside the module and is used to enhance information hiding. 

The most basic form of export is the {\tt public} modifier in Java. Only
public modified classes, methods or fields are available outside a java package.

OSGi bundles and JSR 277 modules allow a user to specify a set of package imports 
and exports that specifies the modules required packages, and the set of packages 
it makes available outside the module. They also both allow a module
to explicitly import another bundle/module and its packages.

Component Nextgen \cite{componentnextgen} uses interfaces for modules, which
specify a parameterized signature of classes that a module provides.

\subsection{Instantiation}
With increasingly large and complex systems that involve a large number of components,
each possibly requiring different versions of a common module,
the ability to allow multiple versions of the same class to coexist become necessary.
Indeed, a single module may require two versions of a another module because its
components require different versions. This implies the creation of multiple instances 
of a module in a repository from which the classes are to be loaded, and the ability
for a module to load and distinguish multiple versions of that module in its code.

iJAM \cite{iJAM} provides a module instantiation feature that allows a module to create
separate instances of its imported modules. This was a solution for type lookup
problems caused by the {\it parent-then-self} class loading scheme proposed for
JSR 294 \cite{JSR294}.

However, type lookup when there are multiple instances of the same module are still
reliant of fixed lookup orders: {\it self-then-parent} for iJAM and the reverse for JSR 294.
Real world applications require more flexibility than is provided by a single fixed
lookup rule \cite{iJAMComments}.