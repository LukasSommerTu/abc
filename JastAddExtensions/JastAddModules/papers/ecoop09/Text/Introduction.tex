Module systems for Java and similar languages have been designed 
to deal with the shortcomings of the package as a modularity construct
and the deeper problem of JAR hell, that is, the problems caused
by wrong class lookup when there are multiple versions of a library
in the classpath. Several module systems for Java-like languages 
have been proposed \cite{javajars, OSGi4, netassemblies, JSR294, JSR277}
that address these problems, and the design of these systems have
revolved around the premise of a module as a container for classes
with additional meta-data to distinguish different versions.

However, the existing module systems still have limitations. Split packages, 
limited ability to express backward compatibility and extension
features are common to the current module systems.

Split packages \cite{iJAMComments} occur when 
a package and its classes occur in more than one visible module. This
causes the same ambiguous class lookup problems caused by JAR hell.
This is a symptom of the lack of ability to specify module sharing
and use.

While substantial support for version constraints are supported
for a module's clients at the point of import, current module 
systems do not have a way for module providers to specify backward 
compatibility (or lack thereof) in the module specification. 
This information is usually found in release documentation and 
it becomes the responsibility of a module's clients, who have comparatively
less information than the module's providers, to update 
their applications based on this information. 

Existing module systems also have limited extension capabilities. Extending
a module's functionality without access to its source code usually 
requires predefined extensibility support coded in by the module's 
providers, usually in the form of access to the points of a class' 
instantiation. Without such explicit support, a client's only option
is usually reimplementation of the parts that contain the instantiation, 
and the use of wrapper classes to reuse and extend behavior. This also
extends to patching, where most often the only option is releasing the 
entirety of a module including the changes, or a destructive update
on the existing module.

We propose that these problems can be solved having object-oriented relationships
between modules akin to those found between classes. Split packages can 
be handled by explicit definitions of module instance sharing and use.
Module extension can be modeled by submodules, and version constraints
by module interfaces. 

The paper is organized as follows. First we describe the
usage and shortcomings of existing module systems in section \ref{moduleops}. We then 
define a module system that contains relations and operations 
that closely resemble object-oriented types in section \ref{moduletypes}. 
In section \ref{eval} we demonstrate that the module
system is able to express existing usage of modules while providing new
capabilities to handle split packages, backwards compatibility and module extension. The module system
itself is implemented in a compiler made using the JastAdd\cite{jastadd} compiler construction
framework, and is demonstrated on a small case study on JHotdraw 7.1 \cite{jhotdraw}.
To demonstrate that the module system contains the functionality provided by existing
module systems, we provide a translation from OSGi bundles, a module system in widespread
use for Java, into the proposed module system in section \ref{translation}.
A short description of the implementation is included in section \ref{implementation}.
The paper then ends with sections on related work and conclusions.
