

%Versioning is done through constraints, usually flat

%Extension is specified without reference to versioning

%Limited and cumbersome support for constraints at the publisher side

%

The contributions of this paper are:

\begin{enumerate}
\item  The definition of object-oriented relations and operations on modules to 
handle extension and versioning;
\item The implementation of the module system as a Java extension and a
small case study to demonstrate the system's features;
\item A translation from OSGi bundles to the proposed system's modules to
demonstrate that the proposed system can express OSGi's bundle composition constraints.
\end{enumerate}

The paper is organized as follows. First we describe the
usage and shortcomings of existing module systems in section \ref{moduleops}. 
We then follow with an example-driven description of the proposed module system in section \ref{moduletypes}. 
In section \ref{eval} we demonstrate that the module
system is able to express existing usage of modules while providing new
capabilities to handle module replacement and merging. The module system
itself is implemented in a compiler made using the JastAdd\cite{jastadd} compiler construction
framework, and is demonstrated on a small case study on JHotdraw 7.1 \cite{jhotdraw}.
To further demonstrate that the module system contains the functionality provided by existing
module systems, we provide a translation from OSGi bundles, a module system in widespread
use for Java, into the proposed module system in section \ref{translation}.
A short description of the implementation is included in section \ref{implementation}.
The paper then ends with sections on related work and conclusions.
