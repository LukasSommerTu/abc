\documentclass{article}

\usepackage{amsmath}
\usepackage{xspace}
\usepackage{xcolor}
\usepackage{stmaryrd}
\usepackage{algorithm}
\usepackage{algorithmic}
\usepackage{listings}
\usepackage{supertabular}

\title{Specifications of Implemented Refactorings}
\author{Max Sch\"afer}

\newcommand{\refactoring}[1]{\textsc{#1}}
\newcommand{\type}[1]{\ensuremath{\text{\textsl{#1}}}}
\newcommand{\util}[1]{\ensuremath{\text{\texttt{#1}}}}
\newcommand{\orelse}{\,\textbf{or}\,}
\newcommand{\assert}{\textbf{assert}\,\,}
\newcommand{\locked}[1]{\ensuremath{\llbracket #1\rrbracket}}
\newcommand{\option}[1]{\ensuremath{\text{\texttt{option}\xspace #1}}}
\newcommand{\sourcelink}[1]{\texttt{#1}}

\definecolor{KWColor}{rgb}{0.5,0,0.67}

\lstset{
  language=[JastAdd]Java,
  basicstyle=\ttfamily\small,
  commentstyle=\footnotesize\rmfamily\emph,
  keywordstyle=\bf\ttfamily\small\color{KWColor},
  morekeywords={with},
  escapeinside={/*@}{@*/},
  literate={[}{{$\lfloor$}}1 {]}{{$\rfloor$}}1,
}

\newcommand{\code}[1]{\lstinline$#1$}
\newcommand{\progoutput}[1]{\texttt{#1}}
\lstnewenvironment{java}{}{}
\newcommand{\kw}[1]{\textbf{\color{KWColor}{#1}}}

\begin{document}
\maketitle

This document collects the pseudo-code specifications of all refactoring implemented in our engine.

\section{Pseudocode Conventions}
We give our specifications in generic, imperative pseudocode. Parameters and return values are informally typed, with syntax tree nodes having one of the types from Fig.~\ref{fig:node types}. Additionally, we use an ML-like \util{option} type with constructors \util{None} and \util{Some} for functions that may or may not return a value.

The names of refactorings are written in \refactoring{small caps}, whereas utility functions appear in \util{monospace}. A list of utility functions with brief descriptions appears in Fig.~\ref{fig:utility}.

Where convenient, we make use of ML-like lists, with list literals of the form $[1; 2; 3]$ and $|xs|$ indicating the length of list $xs$. We also use the higher-order function \util{map}, with lambda expressions to denote the function being mapped over the list. The notation $\locked{d}$ denotes a locked name that binds to declaration $d$. 

Creation of a node is denoted by $\type{NodeType}(a_1,\ldots,a_n)$, where \type{NodeType} is the type of the node being created and $a_i$ are child nodes or other arguments.

\section{The Refactorings}

\subsection{\refactoring{Convert Anonymous to Local}}
This refactoring converts an anonymous class to a local class. Implemented in \sourcelink{TypePromotion/AnonymousClassToLocalClass.jrag}.

\begin{algorithm}
\caption{$\refactoring{Convert Anonymous to Local}(A : \type{AnonymousClass}, n : \type{Name}) : \type{LocalClass}$}
\begin{algorithmic}[1]
\REQUIRE Java
\ENSURE Java $\cup$ locked names
\medskip
\STATE $c \leftarrow \util{getClassInstanceExpr}(A)$
\STATE $s \leftarrow [\refactoring{Extract Temp}](c, \util{unCapitalise}(n))$
\STATE $b \leftarrow \util{enclosingBodyDecl}(s)$
\STATE $\util{lockTypeNames}(b, n)$
\STATE $t \leftarrow \util{asNamedClass}(A, n)$
\STATE $\util{removeTypeDecl}(c)$
\STATE $\util{setTypeAccess}(c, \locked{t})$
\RETURN $\util{insertLocalClass}(s, t)$
\end{algorithmic}
\end{algorithm}

We first retrieve the class instance expression $c$ of which $A$ is a part. Then we apply the \code{Extract Temp} refactoring to move $c$ into its own statement. All references to types named $n$ are locked within the enclosing body declaration $b$. Then $A$ is converted into a class $t$ with name $n$. We remove $A$ from $c$, make sure that $c$ constructs an object of type $t$, and insert $t$ as a local class right before the statement containing $c$.

\subsection{\refactoring{Convert Local to Member Class}}
This refactoring converts a local class to a member class. Implemented in \sourcelink{TypePromotion/LocalClassToMemberClass.jrag}.

\begin{algorithm}
\caption{$\refactoring{Convert Local to Member Class}(L : \type{LocalClass}) : \type{MemberType}$}
\begin{algorithmic}[1]
\REQUIRE Java
\ENSURE Java $\cup$ locked names
\medskip
\STATE $h \leftarrow \util{enclosingType}(L)$
\STATE $\util{closeOverTypeVariables}(L)$
\STATE $\util{closeOverLocalVariables}(L)$
\IF{$\util{inStaticContext}(L)$}
  \STATE $\util{addModifier}(L, \text{\code{static}})$
\ENDIF
\STATE $\util{lockTypeNames}(\util{programRoot}(), \util{name}(L))$
\STATE $\util{lockNames}(L)$
\STATE $\util{removeStmt}(L)$
\RETURN $\util{insertMemberType}(h, L)$
\end{algorithmic}
\end{algorithm}

\subsection{\refactoring{Convert Anonymous to Nested}}
This refactoring converts an anonymous class to a member class. Implemented in \sourcelink{TypePromotion/AnonymousClassToMemberClass.jrag}; see Algorithm~\ref{alg:ConvertAnonymousToNested}.

% Still missing makeFinal, makePublic

\begin{algorithm}[p]
\caption{$\refactoring{Convert Anonymous to Nested}(A : \type{AnonymousClass}, n : \type{Name}) : \type{MemberType}$}\label{alg:ConvertAnonymousToNested}.
\begin{algorithmic}[1]
\REQUIRE Java
\ENSURE Java
\medskip
\STATE $L \leftarrow \refactoring{Convert Anonymous to Local}(A, n)$
\RETURN $\refactoring{Convert Local to Member Class}(L)$
\end{algorithmic}
\end{algorithm}

Note: the implementation additionally handles the case where $A$ occurs in a field initialiser.


\section{Node Types}
\begin{center}
\bottomcaption{Node Types}\label{fig:node types}
\begin{supertabular}{|l|l|}
\hline
\textbf{Node Type} & \textbf{Description} \\ \hline\hline
\type{AnonymousClass} & anonymous class declaration; is an \type{Expr} \\
\type{Assignment} & expression statement consisting of a simple assignment; is a \type{Stmt} \\
\type{Block} & block of statements; is a \type{Stmt} \\
\type{Class} & class declaration; is a \type{Type} \\
\type{Callable} & either a method or a constructor \\
\type{Expr} & expression \\
\type{Field} & field declaration \\
\type{LocalClass} & local class declaration, contains a \type{Class}; is a \type{Stmt} \\
\type{LocalVarDecl} & local variable declaration, part of a \type{LocalVarDeclStmt} \\
\type{LocalVarDeclStmt} & local variable declaration statement; is a \type{Stmt} \\
\type{MemberType} & type declared inside another type; is a \type{Type} \\
\type{Method} & method declaration; is a \type{Callable} \\
\type{MethodCall} & method call \\
\type{Parameter} & parameter declaration \\
\type{Return} & return statement; is a \type{Stmt} \\
\type{Stmt} & statement \\
\type{SuperCall} & super call of a method; is a \type{MethodCall} \\
\type{Type} & type declaration \\
\type{VirtualMethod} & non-\code{private} instance method; is a \type{Method} \\
\type{With} & \code{with} construct (language extension) \\
\hline
\end{supertabular}
\end{center}

We also use the non-node type \type{Name} to represent names.


\section{Utility Functions}
See Fig.~\ref{fig:utility}.

\begin{figure}[hb]
\begin{center}
\begin{tabular}{|l|p{5.3cm}|}
\hline
\textbf{Name} & \textbf{Description} \\ \hline\hline
$\util{below}(n)$ & returns the set of all nodes below $n$ in the syntax tree \\
$\util{calls}(m)$ & returns all calls that may dynamically resolve to method $m$; can be a conservative over-approximation \\
$\util{hostPkg}(e)$ & returns the package of the compilation unit containing $e$ \\
$\util{hostType}(e)$ & returns the closest enclosing type declaration around $e$ \\
$\util{lockMethodCalls}(n)$ & locks all calls to methods named $n$ anywhere in the program \\
$\util{lockNames}(n)$ & locks all names anywhere in the program that refer to a declaration with name $n$ \\
$\util{name}(e)$ & returns the name of program entity $e$ \\
$\util{uses}(m)$ & returns all calls that statically bind to method $m$ \\
\hline
\end{tabular}
\end{center}
\caption{Utility Functions}
\label{fig:utility}
\end{figure}


\end{document}
