\documentclass[12pt,a4paper]{report}

\usepackage{version}
\usepackage{verbatim}

\includeversion{extension:private-pointcut-vars}

\newcommand\extensionbegin[1]{%
\begin{extension:#1}
\hrule
\noindent
If the {\em #1} extension is enabled:

\noindent
}
\newcommand\extensionend[1]{%
\hrule
\end{extension:#1}}

\newcommand\abcdistinctionbegin[0]{%
\hrule
\noindent
The following is specific to abc:

\noindent
}
\newcommand\abcdistinctionend[0]{%
\hrule
}

\begin{document}

\chapter{Aspects}

\abcdistinctionbegin
The source code of all aspects must be available at compile time.
\abcdistinctionend

\section{Weavable classes}

\chapter{Patterns}

\chapter{Static crosscutting}

\chapter{Join points and pointcuts}
\section{Join points}

\extensionbegin{private-pointcut-vars}
The {\em cast} join point occurs every time a value is cast from one type
to another. This includes both reference and primitive type casts.
\extensionend{private-pointcut-vars}

\section{Join point shadows}

\section{Pointcut language}

\subsection{Inspecting and binding values from the context}

If the static type of the object being inspected is equal to or a subtype of
the requested type, then a null value will cause the match to succeed.
If this is not the case, the a null value will cause the match to fail.

Explanation: In the former case, there is no runtime check, and so any
runtime value including null will be accepted. In the latter case a runtime
check is done using the \verb|instanceof| operator, and this returns false
if the argument is null.


\subsection{If}
Expressions inside an if pointcut should not make use of side effects. 
The order of evaluation of such expressions is undefined; as is the number
of times they will be evaluated at any given join point shadow.

\chapter{Declaring new warnings and errors}

\chapter{Advice}

\chapter{Aspect inheritance}
Aspects may be declared abstract. Any aspect may extend either a class
or an abstract aspect, using the {\bf extends} keyword as with classes.
It is not legal for an aspect to extend a non-abstract aspect, or
for a class to extend any kind of aspect.

An abstract aspect may declare abstract pointcuts, using the following
syntax:

\begin{verbatim}
abstract pointcut pc(int x);
\end{verbatim}

Advice in an abstract
aspect may refer to abstract pointcuts. An concrete 
aspect which extends an abstract
aspect must provide an implementation for the abstract pointcuts of its
base aspect. An abstract aspect which extends another abstract aspect
may choose to provide such an implementation; if it does not, then 
such pointcuts are also treated as abstract pointcuts of the extending aspect.

Advice in abstract aspects has no direct effect. However, the advice
will be applied once for each concrete aspect that extends the abstract 
aspect, either directly or indirectly. Any direct or indirect 
references to abstract pointcuts will be resolved in the concrete aspects.

\chapter{Non-standard aspect instantiation}

If no instantiation clause is specified, then it is inherited from the parent
aspect if there is one, or defaults to \verb|issingleton| otherwise.

\chapter{Declare soft}

\chapter{Precedence}
\section{Aspect precedence}
If not overridden by a \verb|declare precedence| declaration, an aspect 
has precedence over any aspect it extends.

\section{Advice precedence}
Advice precedence affects the order in which multiple pieces of advice
at a given join point are executed. A piece of advice may either have 
{\em higher precedence} than another piece of advice, 
{\em lower precedence} than it, 
no precedence relationship with it, 
or be in {\em precedence conflict} with it.

If two pieces of advice in precedence conflict might apply at the same
join point shadow, a compile-time error occurs.

It is also possible for {\em precedence circularities} to exist --- for example
A might have higher precedence than B which has higher precedence than C
which has higher precedence than A. If such a circularity exists between
any of the advice that might apply at a given shadow, then a compile-time
error also occurs.

Conceptually, advice is woven from inside to out, starting with the lowest
precedence advice. The practical effect of this is that higher precedence
before or around advice will start to execute before lower precedence before
or around advice, and that lower precedence around or after advice will 
finish executing before higher precedence around or after advice. 

--- Discuss abnormal termination here and how it can be intercepted with
after throwing advice - but not stopped, except by throwing a different
exception

If two pieces of advice are defined 
in different aspects, then their precedence is defined by the precedence
of those aspects.

If two pieces of advice are defined in the same aspect, and either is after
advice, then the piece of advice that appears later in the aspect has 
precedence over the one that appears earlier. If neither is after advice,
the one appearing earlier has precedence.

If the same piece of advice applies multiple times at one join point
(this can happen if multiple concrete aspects extend the abstract aspect
containing the advice), the order in which they are executed is undefined.

\end{document}