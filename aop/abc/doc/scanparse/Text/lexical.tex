The lexical analysis of AspectJ is complicated by the fact that there are
really three different languages being parsed: (1) normal Java code,
(2)aspect declarations, and 
(3) pointcut definitions.
Each of these three sub-languages has its own lexical structure,  and it 
simplifies the subsequent design of the grammar if the scanner has different 
states and rules for each sub-grammar.

\subsection{Nested Lexical Scopes}
From a conceptual point of view we can think of an AspectJ program 
consisiting of nested lexical scopes.   There are four kinds of lexical
scopes which we refer to by the mode names, {\sc Java}, {\sc Aspect},
{\sc Pointcut} and {\sc PointcutIfExpr}.   
Figure \ref{FIG:LexExample} shows an example of all four kinds of scopes
and each are discussed in more detail in the subsequent subsections.   

\begin{figure}[htbp]
\begin{center}
\begin{footnotesize}
\framebox{
\begin{minipage}{6in}
\begin{java}
\\
{\bf import} java.lang.*;\\
\\
{\bf class} OrdinaryJavaClass \{\\
\hspace{1.0em}{\bf public} {\bf int} x;\\
\hspace{1.0em}{\bf public} {\bf int} y;\\
\\
\hspace{1.0em}String foo({\bf int} x) \\
\hspace{2.5em}\{ {\bf return} ("The value of x + 1 is " + (x + 1));\\
\hspace{2.5em}\}\\
\}\\
\\
{\bf privileged} {\bf aspect} {\textit{/* a privileged aspect with a per declaration in the header */}}\\
\end{java}

\framebox{
\begin{minipage}{4.7in}
\begin{java}
\\
\hspace{1.0em}OrdinaryAspect {\bf percflow} \framebox{ ( {\bf call}({\bf void} Foo.m()) )} \\
\hspace{1.0em}\{ \\
\hspace{2.0em}{\textit{/* declare declaration */}}\\
\hspace{2.0em}{\bf declare} \framebox{{\bf warning}: {\bf call}(*1.Foo+.{\bf new}(..)): \textit{{``pkg ending in 1, class or subclass of Foo"}};}\\
\\
\hspace{2.0em}{\textit{/* pointcut declarations */}}\\
\hspace{2.0em}{\bf pointcut} \framebox{notKeywords(): {\bf call}({\bf void} *if*..*while*({\bf int},{\bf boolean},*for*));}\\
\\
\hspace{2.0em}{\bf pointcut} \framebox{hasSpecialIf(): {\bf if}\framebox{(Tracing.isEnabled())};}\\
\\
\hspace{2.0em}{\textit{/* advice declaration */}}\\
\hspace{2.0em}{\bf after}\framebox{(Point p) {\bf returning}({\bf int} x): {\bf target}(p) \&\& {\bf call}({\bf int} getX())} \\
\hspace{3.0em}\{\\
\hspace{4.0em}System.out.println(\textit{{``Returning int value "}} + x + \textit{{`` for p = "}} + p);\\
\hspace{3.0em}\}\\
\\
\hspace{2.0em}{\textit{/* inter-type member declaration */}}\\
\hspace{2.0em}{\bf int} OrdinaryJavaClass.incr2({\bf int} i)\\
\hspace{3.5em}\{ {\bf return}(x+2);\\
\hspace{3.5em}\} \\
\\
\hspace{2.0em}{\textit{/* ordinary Java declarations */}}\\
\hspace{2.0em}{\bf int} x;\\
\\
\hspace{2.0em}{\bf static} {\bf int} incr3({\bf int} x)\\
\hspace{3.0em}\{ {\bf return}(x+3);\\
\hspace{3.0em}\}\\
\\
\hspace{2.0em}{\textit{/* a nested class */}}\\
\hspace{2.0em}{\bf class} \\
\end{java}

\framebox{
\begin{minipage}{4.5in}
\begin{java}
\\
\hspace{3.0em}NestedClass \{\\
\hspace{3.0em}{\textit{/* In Java mode, after and before are not keywords. */}} \\
\hspace{4.0em}{\bf public} {\bf int} after;\\
\hspace{4.0em}{\bf public} {\bf int} getBefore()\\
\hspace{4.0em}\{ {\bf return}(OtherClass.before);\\
\hspace{4.0em}\}\\
\hspace{2.5em}\} {\textit{// end of NestedClass }}\\
\end{java}
\end{minipage}
}

\begin{java}
\hspace{1.0em}\} {\textit{// end of OrdinaryAspect}}\\
\end{java}
\end{minipage}
}
\end{minipage}
}
\end{footnotesize}
\end{center}

\caption{AspectJ code with nested lexical scopes \label{FIG:LexExample}}
\end{figure}

\subsubsection{{\sc Java} mode}
The outermost scope is always has
{\sc Java} mode.  In this mode all tokens are scanned exactly as in
Java, with the exception that {\bf privileged},  {\bf aspect} and 
{\bf pointcut} are considered to be keywords and cannot be used an identifiers.

\subsubsection{{\sc Aspect} mode}
Inside the program one can have a nested {\sc Aspect} scope which begins
just after the keyword {\bf aspect} and ends at the end of the aspect's body.
Figure \ref{FIG:LexExample} shows one  {\sc Aspect} scope corresponding
to the body of the declaration of the aspect named "OrdinaryAspect''. 

In {\sc Aspect} mode all symbols are exactly
the same as in {\sc Java} mode,  except for the addition of  keywords
{\bf after}, {\bf around}, {\bf before}, {\bf declare}, 
{\bf issingleton}, {\bf percflow}, {\bf percflowbelow}, {\bf pertarget},
{\bf perthis}, {\bf pointcut}, {\bf proceed}. 
Just like normal Java keywords, these
additional keywords cannot be used as identifiers, inside of an
{\sc Aspect} scope.

\subsubsection{{\sc Pointcut} mode}
Whereas the {\sc Java} and {\sc Aspect} modes are very similar, basically
differing only on the keywords recognized, the {\sc Pointcut} mode
has a completely different lexical structure.   
In Figure \ref{FIG:LexExample},  the lexical scopes for pointcuts are
shown by the boxes nexted inside the Aspect.   There are four contexts in
which pointcut scopes occur, as follows:

\begin{description}
\item[Pointcut Context \#1 - a Per clause in an aspect declaration:]
The header of an aspect
declaration ends with an an optional per clause.  A per clause consists
of either the keyword {\bf issingleton} or  a parenthesized pointcut
expression preceeded by one of the kewords {\bf percflow}, 
{\bf percflowbelow}, {\bf pertarget} or {\bf perthis}.    
A {\sc Pointcut} scope starts after one of the per keywords and ends at
the matching closing parenthesis that surrounds the pointcut.   Figure
\ref{FIG:LexExample} shows a nested scope for a pointcut expression 
following the {\bf percflow} keyword.

\item[Pointcut Context \#2 - body of a {\bf declare} declaration:]
Inside the body of an aspect one can define
a declare declaration.   A lexical {\sc Pointcut} scope begins just after
the keyword {\bf declare} and ends at the {\bf ;} terminating the 
declaration.  

The example program shows a  
declaration of  a {\bf warning} which matches all calls to constructors of
classes found in packages ending in the digit 1, with classname Foo or
a subclass of Foo. 

\item[Pointcut Context \#3 - body of a pointcut declaration:]

Pointcut declarations are provided in AspectJ as a way of 
defining a named pointcut.  In the example program in 
Figure \ref{FIG:LexExample} two such declarations are given, one
for {\em notKeywords} and another for {\em hasSpecialIf}.   
Inside pointcut declarations a pointcut lexical scope begins
immediately following the {\bf pointcut} keyword and ends 
after the {\bf ;} terminating the pointcut declaration.
Pointcut declarations can appear both inside aspects and inside
ordinary Java classes.

\item[Pointcut Context \#4 - header of an advice declaration:]

Advice declarations have a pointcut expression in their header.
All such pointcuts will be preceeded by one of the keywords
{\bf before}, {\bf after} or {\bf around}.    For example, in
Figure \ref{FIG:LexExample},  a pointcut follows an {\bf after} keyword.
The pointcut ends before the body of the pointcut begins, signalled by
a {\bf \{}.

Thus, a pointcut context starts immediately after a {\bf before},
{\bf after} or {\bf around} token, and ends at the first opening
brace encountered.

\end{description}

\subsubsection{{\sc PointcutIfExpr} mode} 

Pointcuts have no lexical scopes nested inside them, 
except for one case, the
{\bf if} pointcut.   The {\bf if} pointcut contains a boolean
expression, which is just Java code.
We define a special
mode called {\sc PointcutIfExpr} which starts right after the
{\bf if} keyword inside a pointcut, and ends at the terminating
parenthesis closing the boolean expression.   The lexical structure,
in terms of tokens recognized, is identical to the {\sc Aspect} mode.    
However, in the implementation of the scanner, the end of 
{\sc PointcutIfExpr} mode always signals a return to 
{\sc Pointcut} mode.

In Figure \ref{FIG:LexExample}, the pointcut declaration named
{\em hasSpecialIf} shows an example of a nested {\sc PointcutIfExpr}
lexical scope.

\subsection{Nested Aspects, Classes and Interfaces}

In AspectJ classes, interfaces and aspects may be nested inside each 
other.   In terms of nested lexical scope,  a new scope is entered 
each time the keywords {\bf class}, {\bf interface} and {\bf aspect}
are entered,  and the scope is exited at the closing right brace.
In the case of {\bf class} and {\bf interface} the scope entered
has {\sc Java} mode,  whereas for the case of {\bf aspect} the scope
entered has {\sc Aspect} mode.

At the bottom of Figure \ref{FIG:LexExample} we give the declaration of
a nested class called {\em NestedClass}.   Note that inside the class
declaration is in {\sc Java} mode,  so the keywords recognized are those
corresonding to {\sc Java} mode (i.e. Java keywords plus {\bf aspect},
{\bf privileged} and {\bf pointcut}).    Thus, in the example,  
{\em before} and {\em after} are considered identifiers, not keywords.   
Note that this use of
inner classes provides a mechanism for referring to variables defined
in other classes that may have the same name as keywords in 
{\sc Aspect} mode.  In our example we have defined the method
{\em getBefore} to read the value of {\em OtherClass.before}.

The above rule for entering a new lexical scope upon encounter of
the keyword {\bf class} is complicated by the fact that {\bf class}
does not always signal a new class declaration in Java. In particular,
it can be used to return a {\em Class} object that represents a type,
as in {\em C.class} (this is useful, for example, to create typed lists, 
where the intended element type is stored with the list structure).
All such uses of the {\bf class} keyword are preceded by a
dot, and class declarations themselves are never preceded by a dot. 
For that reason, the lexer records whether the last emitted token
was a dot; if it is, then the {\bf class} keyword does {\em not}
cause a transition to a new lexical scope.

\subsection{Lexical Structure of Pointcuts}

The language for defining pointcuts is a very special purpose
language that provides a way of specifying identifier patterns,
classname patterns, and more complex expressions involving
patterns.     

\subsubsection{Examples of differences from the Java lexical structure}
The example program clearly demonstrates ways in which the
lexical structure of pointcuts is very different from the 
lexical structure of Java.

For example, if one were to use the ordinary Java
lexical rules, then the expression ``*1.Foo+.new(..)'' 
would be tokenized as:

{\tt[ op("*"), fp\_literal(1.0), Id("Foo"), op("+"), op("."),  
keyword("new"), op("("), op("."), op(".") op(")")]}.  

\noindent
However, in
pointcuts, the intended lexical structure is quite different and
would be tokenized as:

{\tt[ IdPat("*1"), op("."), id("Foo"), op("+"), op("."),
keyword("new"), op("("), op(".."), op(")")]}.   

\noindent 
Note that ``*1" is
an identifier pattern, which matches any identifier ending in 1.  Also, the
sequence ``+." is recognized as one token, not two.  This simplifies
the grammar and avoids shift-reduce conflicts.  Finally, the sequence
``.." is recognized as one token, which also simplifies the grammar.

Another example of the need for a special lexical structure for pointcuts
is given in the definition of the pointcut {\em notKeywords}.   The ordinary
Java lexical rules would tokenize the expression ``*if*..*while*" as:

{\tt[ op("*"), keyword("if"), op("*"), op("."), op("."), op("*"), 
keyword("while"), op("*")]}.

\noindent
whereas this expression inside a pointcut has a completely different lexical
structure, namely:

{\tt[ IdPat("*if*"), op(".."), IdPat("*while*") ] }.
 
\subsubsection{Tokens in pointcuts}
Since the Java lexical structure clearly doen't match the pointcut
language very well,  a completely different lexical structure is
defined for pointcuts.   This can be summarized as follows.

\begin{description}
\item{Keywords:}
All of the keywords in {\sc Java} mode (including
{\bf aspect} and {\bf privileged}), plus the following:
{\bf adviceexecution}, {\bf args}, {\bf call}, {\bf cflow}, 
{\bf cflowbelow}, {\bf error}, {\bf execution}, {\bf get},
{\bf handler}, {\bf initialization}, {\bf parents}, {\bf precedence},
{\bf preinitialization}, {\bf returning}, {\bf set}, {\bf soft},
{\bf staticinitialization}, {\bf target}, {\bf throwing}, 
{\bf warning}, {\bf within}, {\bf withincode}.

Note that extra keywords in {\sc Aspect} mode,  such as {\bf before}, are
not keywords in the {\sc Pointcut} mode, and similarly the extra keywords
in {\sc Pointcut} mode are not keywords in {\sc Aspect} mode.    

\item{Symbols:}
The symbols recognized in pointcuts are: 
{\tt op("(")}, {\tt op(")")}, {\tt op("[")}, {\tt op(".")},
{\tt op(",")}, {\tt op(":")}, {\tt op(";")}, {\tt op("\{")},
{\tt op("\"")}, {\tt op("!")}, {\tt op("\&\&")}, {\tt op("||")},
{\tt op("..")}, {\tt op("+")}. 

\item{Identifiers and Identifier Patterns:}  Identifiers are matched
using the same regular expression as in {\sc Java} mode, namely:

{\tt Identifier = [:jletter:][:jletterdigit:]*}

\noindent
Identifer patterns are recognized as:

{\tt IdentifierPattern = ( "*" | [:jletter:] ) ( "*" | [:jletterdigit:] )*}

Since identifiers and identifer patterns are used in pointcuts to
specify names that may occur anywhere, including Java code that has
been defined in a library,  it is possible that a pattern might want
to refer to something with the same name as one of the extra keywords.
This is handled later by the grammar,  where the extra keywords are
explicitly allowed as one alternative of the rule for 
{\em simple\_name\_pattern},  see Section \ref{SEC:NamePatterns}.
\end{description}


\begin{comment}
\subsection{Implementing the Lexical Analysis}

Polyglot's  JFlex-based  scanner for the base Java langauge has three states:
\begin{description}
\item[{\sc YYInitial}:]  This is the initial state for the scanner, and
it corresponds to all tokens except for strings and character literals.
\item[{\sc String}:]  This state is entered upon recognizing a {\tt "}.
In this state the entire string is scanned and returned as one token.
At the closing {\tt "} character, the state reverts to {\sc YYInitial} and
the string token is returned.
\item[{\sc CharLiteral}:] This state is entered upon recognizing a {\tt '},
the character is scanned, handling escape sequences and then the character
literal token is returned and the state reverts to {\sc YYInitial}.
\end{description}
\end{comment}
